% Poner en perception (sonido, luz). ??

\title{Redundancy in multimedia}
\author{Vicente González Ruiz}
\maketitle

\section{Statistical}
\href{https://en.wikipedia.org/wiki/Redundancy_(information_theory)}{Statistical
  redundancy} is present in all those sequences of symbols where we
can infeer a probability of ocurrence of a symbol taking into
consideration the rest of symbols of a sequence. Statistical redudancy
can be found in text, audio, image and video, among other types of
signals.

Statistical redundancy can be removed by text compressors.

Statistical redundancy is also called source-coding redundancy~\cite{kondoz2009visual}.

\section{Temporal}
Temporal redundancy is shown by sequences of samples when those
samples are similar in value and when paterns of samples tend to
repeat. Temporal redundancy is found in all those time-dependant
signals, such as
\href{https://en.wikipedia.org/wiki/Inter_frame}{audio} and
\href{https://en.wikipedia.org/wiki/Inter_frame}{video}, among others.

Temporal redundancy can be removed by most audio and video codecs.

\section{Spatial (2D)}
\href{https://robbfoxx.wordpress.com/2015/07/12/discussion-6-2-1-what-is-redundancy-temporal-redundancy-and-spatial-redundancy/}{Spatial
  redundancy} is present basically in images, because pixels tend to
be similar to their neighbors or tend to repeat textures. See
Fig~\ref{fig:correlacion_lena}.

\begin{figure}
  \pngfig{graphics/correlacion_lena}{8cm}{800} %
  \caption{Spatial redundancy in images.}
  \label{fig:correlacion_lena}
\end{figure}

%\bibliography{video-compression}
